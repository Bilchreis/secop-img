\usepackage{ifluatex}% Test
\usepackage{amsmath}  %% muss vor fontspec geladen werden

\ifluatex
\usepackage{fontspec}% Vektorschrift
\usepackage{unicode-math}% Mathematikschrift
\usepackage{babel}
\usepackage[german=guillemets]{csquotes}
\usepackage{microtype} % optischer Randausgleich etc.

%\setmathfont[Scale=0.9]{texgyrepagella-math.otf}
%\setmathfont{XITS Math}
\newfontface\Erna[%
Scale=MatchUppercase]{Arial}
\else
\usepackage[T1]{fontenc}%Schriftkodierung
\usepackage{kpfonts}%Kepler Fonts mit mathe
\usepackage[utf8]{inputenc}% Eingabekodierung
\fi

%\usepackage[showframe,nomarginpar, margin=.5in]{geometry}



%\usepackage{showframe}

%\usepackage[%showframe,
%lmargin=2cm,rmargin=2cm,includeheadfoot]{geometry}%Ränder





\usepackage{xspace}%schütze Leerzeichen nach newcommand

%\usepackage{libertine}


%\usepackage{animate}%für animationen

%\usepackage{blindtext}%nur für demo!!!

\usepackage[acronym]{glossaries}



%\usepackage{imakeidx}% für einen Index, erledigt selbst den makeindex-Lauf
%\makeindex[intoc] %\jobname.idx
%\makeindex[name=Person, title=Personenverzeichnis,intoc] % Person.idx












%%%%neue packages
%\usepackage{mathtools}
\usepackage{tcolorbox}
\usepackage{caption}
\usepackage[labelformat=simple]{subcaption}
\renewcommand\thesubfigure{(\alph{subfigure})}
\usepackage{listings}
\usepackage{siunitx}
\usepackage{todonotes}
\usepackage[outdir=./]{epstopdf}
\usepackage{graphicx}
\usepackage{nbaseprt}
\usepackage{colortbl}
\usepackage{setspace}
\usepackage{tikz-uml}






\usepackage{relsize}
\usepackage{tkz-euclide}
\usepackage{pgfplots}
\usepackage{multirow}
\usepackage{csvsimple}
\usepackage{pgfplotstable} % Generates table from .csv
\usepackage{array}
\usepackage{makecell}
\usepackage[ruled, lined, linesnumbered, commentsnumbered, longend]{algorithm2e}
\usepackage[edges]{forest}
\usepackage{listings}
\usepackage{xcolor}
\usepackage{capt-of}

\makeatletter


\newcommand\Wider[2][3em]{%
	\makebox[\linewidth][c]{%
		\begin{minipage}{\dimexpr\textwidth+#1\relax}
			\raggedright#2
		\end{minipage}%
	}%
}



\definecolor{eclipseKeywords}{RGB}{127,0,85}
\colorlet{numb}{magenta!60!black}

\lstdefinelanguage{json}{
	basicstyle=\normalfont\ttfamily,
	commentstyle=\color{eclipseStrings}, % style of comment
	stringstyle=\color{eclipseKeywords}, % style of strings
	numbers=left,
	numberstyle=\scriptsize,
	stepnumber=1,
	numbersep=8pt,
	showstringspaces=false,
	breaklines=true,
	frame=lines,
	backgroundcolor=\color{gray}, %only if you like
	string=[s]{"}{"},
	comment=[l]{:\ "},
	morecomment=[l]{:"},
	literate=
	*{0}{{{\color{numb}0}}}{1}
	{1}{{{\color{numb}1}}}{1}
	{2}{{{\color{numb}2}}}{1}
	{3}{{{\color{numb}3}}}{1}
	{4}{{{\color{numb}4}}}{1}
	{5}{{{\color{numb}5}}}{1}
	{6}{{{\color{numb}6}}}{1}
	{7}{{{\color{numb}7}}}{1}
	{8}{{{\color{numb}8}}}{1}
	{9}{{{\color{numb}9}}}{1}
}


\newcommand{\at}{\makeatletter @\makeatother}






\usetikzlibrary{%
	arrows,%
	calc,
	shapes,
	arrows,
	shapes.misc,% wg. rounded rectangle
	shapes.arrows,%
	chains,%
	matrix,%
	positioning,% wg. " of "
	scopes,%
	decorations,% /pgf/decoration/random steps | erste Graphik
	shadows,%
	3d,%
	patterns,%
	patterns.meta,%
	automata,%
	backgrounds,%
	intersections,%
	tikzmark,%
	positioning,%
	angles,%
	quotes,%
	calc,
	fit
}



\tikzset{
	box/.style = {
	fill = black!5,
	line width=1mm,
	inner sep=1.5em,
	rounded corners= 1em,
	draw=black,
	fill opacity=0.5,
	draw opacity = 0.7	
	}
}

\tikzset{% UML2 Activity Diagram Styles
	caption/.style    = {node distance=1em},
	process/.style    = {fill=black!5,draw, thick, rounded corners=0.8em, minimum height = 3em,
		minimum width = 5em, align=center, inner sep=1em,node distance=1em},
	object/.style    = {fill=black!5,draw, thick, rectangle, minimum height = 3em,
		minimum width = 3em, align=center, inner sep=1em,node distance=1em},
	pin/.style    = {fill=black!5,draw, thick, rectangle, minimum height = 0.6em,
		minimum width = 0.6em, node distance=-1pt, inner sep=0,font=\relsize{-3.5}},
	start/.style      = {fill=black,draw,circle,node distance=1em}, 
	group/.style      = {color=black,thin,rounded corners=0.8em, rectangle,fill = black!5}, 
	groupCaption/.style      = {above=0.2cm,very thick,right=0.2cm, fill = white,draw = black,font = \scshape}, 
	input/.style    = {coordinate,node distance=1.5em}, 
	output/.style   = {coordinate,node distance=1.5em}, 
	between/.style args={#1 and #2}{ % http://tex.stackexchange.com/a/138828/15602
		at = ($(#1)!0.5!(#2)$)
	},
	notefield/.style    = {fill=green!5,draw, thick, minimum height = 3em,
		minimum width = 3em, align=left, inner sep=0.5em,node distance=3em,,text width=4cm},
	loop/.style = { rounded corners=0.8em,dashed,rectangle split, rectangle split,
		rectangle split parts=3, very thick,draw=black,  align=center,minimum height = 4cm,rectangle split part align={center, left, left}},
	fitting node/.style={
		inner sep=0pt,
		fill=none,
		draw=none,
		reset transform,
		fit={(\pgf@pathminx,\pgf@pathminy) (\pgf@pathmaxx,\pgf@pathmaxy)}
	},
	reset transform/.code={\pgftransformreset}
	
}



\tikzset{
	base/.style    = {
		rectangle,
		draw,
		anchor=west,
		minimum height=0.8cm,
		minimum width=1.6cm,
		fill=white,
		drop shadow={opacity=0.5,fill=black},
		node distance = 7cm
	},%
	node/.style = {
		base,
		fill = \nodecol,
	},%
	ecs/.style = {
		base,
		fill=\ecsscol
	},%
	active/.style = {
		rectangle,
		draw,
		minimum width = 0.2cm,
		minimum height=1cm,
		fill = gray!20,
		node distance = 1mm and 1cm
	},%
	lifeline/.style ={
		line width=0.3mm,
		-,
		color = black,
		line cap = round,
		dash pattern=on 0pt off 2.5pt
	},%
	secop/.style ={
		line width=0.5mm,
		color = black
	},%
	synchron/.style ={
		->,
		>=triangle 60
	},%
	asynchron/.style ={
		->,	
		>=angle 60
	},%
	syncreturn/.style ={
		->,	
		dashed,
		>=angle 60
	},%
	mess/.style ={
		->,
		>=angle 60
	},%
	messtext/.style ={
		font= \scriptsize\ttfamily,
		midway,
		above,
		sloped,
		align = center,
		text width = 7.5cm 
	},%
	blockstyle/.style={
	anchor = north west,
	font =\ttfamily
	}
}

\tikzset{
	statbar/.style    = {
		font = \scriptsize,
		rectangle,
		draw,
		thin,
		minimum width=0.4cm,
		node distance = 0cm,
		opacity = 0.4,
		inner sep = 2pt
	},%
	statidle/.style    = {
		statbar,
		fill = idle,
		draw = idle
	},%
	statbusy/.style    = {
		statbar,
		fill = busy,
		draw = busy
	},%
	statundefined/.style    = {
		statbar,
		pattern=north west lines,
		pattern color=gray,
		draw = gray
	}%
}


\definecolor{devicecol}{HTML}{fc8d59}
\definecolor{signalcol}{HTML}{91bfdb}
\definecolor{signalxcol}{HTML}{ffffbf}

\def\devcol{devicecol!60}
\def\sigcol{signalcol!60}
\def\sigxcol{signalxcol!60}


\newcommand{\drawll}[2]{
	\draw[lifeline] (#1) -- ([yshift=- #2] #1);
}


\newcommand{\makeGroup}[6]{
	\draw [group,fill= #6](#2-0.5,#3+1.5-#5)rectangle(#2+#4+0.5,#3+1.5);
	\node at (#2,#3+1.5) [groupCaption] {#1};
}



%### DEFINITIONS

%\makeGroup{Title}{Top}{Left}{Width}{Height}
